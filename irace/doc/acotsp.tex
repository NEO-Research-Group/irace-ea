\subsection{ACOTSP: Tuning for solution quality}

ACOTSP~\cite{Stu2002} is a software package that implements various ant 
optimization algorithms to tackle the symmetric traveling salesman problem.

We consider in this section the tuning of all its $11$ parameters.
In particular, we present the tuning of ACOTSP as a case study of using
\irace for tuning algorithms for optimization problems. We are interested
in a parametrization of ACOTSP that finds the best possible solution
quality for some TSP instances in a given time. 
Therefore, the computation effort that can be spent for the tuning is defined
by the number of experiments \parameter{maxExperiments} (we set this option 
to $3000$). The setup
used for the tuning of ACOTSP is summarized in 
Table~\ref{tab:acotsp_tuning_conf}. We use a training set of $1000$ 
instances and a testing set of $300$ instances, all of them having
$750$ cities.

\begin{table}[th]
  \centering
  \caption{Scenario setup for tuning ACOTSP}
  \label{tab:acotsp_tuning_conf}
\begin{tabular}[t]{rr}
\toprule
Max number of experiments (\parameter{maxExperiments}) & 3000 \\
Time per experiment & 5 seconds\\
\bottomrule
\end{tabular}
\end{table}

Note that the number of experiments is a parameter of \irace, but the
time used by the ACOTSP algorithm for each experiment is a fixed
parameter, and thus we set it up in the target runner file defined by 
\parameter{targetRunner}. A second possibility would be to put this parameter 
in the parameter file with a single possible value, and this would not 
affect the tuning of the candidate parameters. 

As examples of target runners and parameter files we provide the necessary files for 
tuning ACOTSP in the directory \texttt{examples/acotsp}.






% LocalWords:   softRestart maxExperiments
%%% Local Variables: 
%%% mode: latex
%%% TeX-master: "documentation"
%%% End: 

